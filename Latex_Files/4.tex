\section{Significance of $\beta$ and $R^2$ Values}
The results of the CAPM regression, displayed in Table~\ref{tab:equity_results}, confirm the model's ability to explain the 
returns of many stocks. 
Stocks like Johnson \& Johnson ($\beta$=0.752, $R^2$=0.458) and Pfizer ($\beta$=0.953, $R^2$=0.449) exhibit significant
$\beta$ coefficients and high $R^2$ values, indicating strong sensitivity to market movements. 
These results validate the CAPM model's assumption that systematic risk is the primary driver of returns for these stocks.
However, stocks with lower $R^2$ values, such as Cigna and Labcorp Holdings, suggest the presence of significant idiosyncratic 
risks or external factors that are not captured by the CAPM framework.
These findings emphasize the model's limitations when applied to companies influenced by unique events or niche market 
dynamics.
To improve the model's accuracy for such cases, additional factors, such as industry-specific risks or macroeconomic
variables, could be incorporated. 
This would help account for variability that the CAPM model alone cannot explain, particularly for stocks with lower market 
correlations.

\begin{table}
    \centering
    \renewcommand{\arraystretch}{1.2} % Adjust row height
    \footnotesize
    \begin{tabular}{|l|c|c|c|c|c|c|}
        \hline
        \rowcolor{unired!30} \textbf{Equity} & \textbf{$\beta$} & \textbf{$\beta$ p-value} & \textbf{White p-value} & \textbf{BG p-value} & \textbf{$R^2$} & \textbf{HAC p-value} \\ \hline
        Johnson \& Johnson & 0.752 & 0 & 7.19E-10 & 0.178 & 0.458 & 0.001 \\ \hline
        \rowcolor{gray!10} Boston Scientific & 0.91 & 0 & 2.17E-05 & 0.846 & 0.21 & 0.003 \\ \hline
        Eli Lilly & 0.942 & 0 & 0.013 & 0.085 & 0.346 & 0.009 \\ \hline
        \rowcolor{gray!10} Pfizer & 0.953 & 0 & 0.209 & 0.094 & 0.449 & 0.008 \\ \hline
        Teleflex & 0.968 & 0 & 0.921 & 0.762 & 0.3 & 0.010 \\ \hline
        \rowcolor{gray!10} Cigna & 1.056 & 0 & 0.119 & 0.988 & 0.214 & 0.015 \\ \hline
        Revvity & 1.218 & 0 & 0.130 & 0.999 & 0.265 & 0.020 \\ \hline
        \rowcolor{gray!10} Medtronic & 0.859 & 0 & 0.477 & 0.192 & 0.384 & 0.012 \\ \hline
        Labcorp Holdings & 0.921 & 0 & 0.207 & 8.85E-07 & 0.236 & 0.005 \\ \hline
        \rowcolor{gray!10} Humana & 1.081 & 0 & 2.87E-05 & 0.989 & 0.235 & 0.014 \\ \hline
    \end{tabular}
    \caption{Regression results for each equity, including Beta, p-values, and various test results.}
    \label{tab:equity_results}
\end{table}

\section{Heteroscedasticity Issues}
The White Test identifies heteroscedasticity in stocks such as Johnson \& Johnson and Humana, indicating non-constant residual variance. This violation of CAPM assumptions suggests that the reliability of the regression results may be compromised, as standard errors may be underestimated or overestimated.
To address this issue, robust standard errors (HAC) were applied, which adjust for heteroscedasticity and provide more reliable parameter estimates. These adjustments ensure that the significance levels of $\beta$ coefficients remain valid despite violations of homoscedasticity.
Future analyses could explore whether specific periods or events contribute to heteroscedasticity. Understanding these patterns may help refine the CAPM model's application to stocks affected by irregular variance in their residuals.

\section{Autocorrelation in Residuals}
The Breusch-Godfrey Test detects autocorrelation in stocks such as Labcorp Holdings, suggesting that the residuals are not 
independently distributed.
This violates the CAPM assumption of no autocorrelation, which can lead to biased parameter estimates and reduced model 
reliability.
To mitigate the impact of autocorrelation, adjustments such as incorporating lagged variables or alternative error correction
methods can be applied. 
These techniques improve the robustness of the regression results and enhance the reliability of the CAPM model for stocks 
with autocorrelated returns.
Despite these challenges, the CAPM model remains effective for analyzing diversified portfolios.
The reduction of idiosyncratic risks and the alignment of residuals with model assumptions further support its applicability 
at an aggregate level.
However, for individual stocks with significant deviations, additional diagnostic tools and model refinements are necessary to 
ensure accurate analysis.