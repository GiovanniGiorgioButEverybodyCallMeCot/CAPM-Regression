This section validates the reliability of the CAPM model through diagnostic tests. 
These tests aim to ensure that key assumptions, such as constant residual variance (homoscedasticity) and the absence of 
autocorrelation, are satisfied. When violations are detected, adjustments, such as robust standard errors, are applied to 
improve the reliability of the results.
The results confirm that the CAPM model effectively explains the returns of many stocks in the Healthcare sector. 
For example, Johnson \& Johnson ($\beta$=0.752, p<0.01) and Pfizer ($\beta$=0.953, p<0.01) show significant $\beta$
coefficients, indicating their sensitivity to market movements.
High $R^2$ values suggest that a substantial portion of the return variability is explained by the model.
However, for stocks like Cigna and Labcorp Holdings, lower $R^2$ values point to the potential influence of idiosyncratic 
factors not captured by the CAPM.
The diagnostic tests highlight some limitations. For instance, the White Test identifies heteroscedasticity in stocks such as 
Johnson \& Johnson and Humana, indicating that residual variance is not constant. 
To address this issue, we applied robust standard errors (HAC). 
Additionally, the Breusch-Godfrey Test reveals autocorrelation in the residuals for Labcorp Holdings, requiring further 
adjustments to ensure the validity of the model. 
Despite these challenges, the F-statistics and regression p-values confirm the overall significance of the model.
The analysis of the equally weighted portfolio further supports these results. 
Diversification reduces idiosyncratic risk and enhances the stability of parameter estimates. 
Compared to individual stocks, the portfolio demonstrates a stronger and more consistent linear relationship with market 
returns, reinforcing the applicability of the CAPM model at an aggregate level.

