\section{Structural Breaks and Chow Test}

In this section, we employ the Chow test to assess whether a significant structural change occurs in the regression models at 
specific points in time. 
To ensure the robustness of our analysis, we first established a minimum data subset size for the unrestricted models, 
setting it to 10\% of the total dataset, in order to maintain statistical validity while preserving sufficient data for
meaningful comparisons.
Subsequently, the Chow test was systematically applied to all linear regressions conducted on the selected equities, 
allowing us to identify potential structural breaks across the dataset, that is, significant changes in the regression
parameters caused by external shocks, market-wide events, or company-specific factors. 
Structural breaks may reflect changes in the relationship between excess returns and market behavior, such as shifts in 
systematic risk $(\beta)$ or unexplained excess returns $(\alpha)$, driven by macroeconomic conditions,
regulatory adjustments, or sector developments.
Structural breaks were identified by analyzing the p-values from the Chow test over time: a p-value below 0.01 indicated a
significant change in CAPM parameters at that point.
In this analysis, only periods of at least two consecutive months with p-values below the threshold were considered as
break dates, ensuring that the identified breaks reflect sustained shifts rather than noise or transient fluctuations.

\begin{figure}[h!]
    \centering
    \includegraphics[width=0.8\textwidth]{images/chowmoving.png}
    \caption{Chow Test performed for all equities in search of structural breaks.}\label{fig:chowmoving}
\end{figure}

The results of this analysis are displayed in Figure~\ref{fig:chowmoving}, highlighting that different equities exhibit 
fundamentally distinct behaviors. 
Notably, certain equities, such as \textit{Eli Lilly}, \textit{Boston Scientific}, and \textit{Teleflex}, show p-values
consistently below the 0.01 threshold for extended periods, indicating prolonged structural breaks; this raises the question
of whether the CAPM relationship for these equities in these periods is truly linear or whether a different modeling approach
may be warranted.

\section{Periods of Shared Structural Breaks}

Despite these observations, no clear or consistent pattern of structural breaks emerges across all equities; to explore 
potential commonalities further, an additional histogram was generated (Figure~\ref{fig:struct_break_freqs}), which illustrates
the frequency and overlap of structural breaks shared among different equities.


\begin{figure}[h!]
    \centering
    \includegraphics[width=0.8\textwidth]{images/struct_break_freqs.png}
    \caption{Frequency of months identified as structural break points.}\label{fig:struct_break_freqs}
\end{figure}

Figure~\ref{fig:struct_break_freqs} highlights two important aspects of the data: 
first, there is no single period that constitute a break date for all equities, the maximum number of companies sharing a 
structural break at any given time is in fact 4; second, the identified break dates align closely with known economic and
market events.
Specifically, the data reveal three distinct periods of significant relevance, indicating concentrated structural breaks 
that may correspond to external shocks or sector-wide disruptions, those periods are:

\begin{enumerate}
    \item  \textbf{2003}: Structural breaks are observed throughout most of the year, coinciding with the SARS epidemic. 
    This period also aligns with economic and legislative developments, including the introduction of the Medicare
    Modernization Act, which reshaped healthcare policy and access in the United States.
    \item \textbf{October and November 2008}: These months mark the onset of the 2008/2009 financial crisis.
    \item \textbf{February and March 2020}: Structural breaks during this time correspond to the emergence of the COVID-19
    pandemic and the widespread implementation of precautionary measures.
\end{enumerate}

An important observation is that the structural breaks identified in 2003 extended over a prolonged period, spanning multiple
consecutive months, whereas the impact of the COVID-19 pandemic appears to have been more concentrated and shorter in duration
(see Figure~\ref{fig:struct_break_freqs}).
The prolonged effect in 2003 could reflect the gradual adjustment of the healthcare market to the SARS epidemic and concurrent
economic and legislative changes.
In contrast, the shorter duration of structural breaks during COVID-19 may indicate that advancements in epidemic preparedness,
such as improved healthcare infrastructure, rapid vaccine development, and the implementation of targeted government 
interventions, helped mitigate the impact of the pandemic on the healthcare market. 
Alternatively, the healthcare market may have simply grown so significantly in size and diversification over time that it 
developed a level of resilience that enables it to absorb and adapt to substantial disruptions in the economic landscape.
This explanation also accounts for the observation that, according to the Chow test, the majority of equities were not
significantly impacted during these events: the increased resilience and diversification of the healthcare market may have
allowed many companies to maintain stability despite the broader economic disruptions.
While these hypotheses provide potential explanations for the observed differences, a thorough investigation into these claims
is beyond the scope of this analysis.


\section{Chow Test on Portfolio}

In order to further investigate the impact of significant events in the economic environment, the same Chow test was 
performed on the excess returns of the portfolio. 
This approach aggregates the behavior of individual equities into a single measure, possibly allowing for the identification 
of systemic disruptions across the healthcare market.

\begin{figure}[h!]
    \centering
    \includegraphics[width=0.8\textwidth]{images/portchowlio.png}
    \caption{Frequency of months identified as structural break points.}\label{fig:portchowlio}
\end{figure}

The results, displayed in Figure~\ref{fig:portchowlio}, show that the Chow test identifies a single time period with
particularly significant structural breaks in the portfolio's return dynamics: the beginning of 2009, likely reflecting the 
delayed impact of the 2008 financial crisis; this suggests that the crisis, while primarily affecting financial markets, 
also influenced the healthcare sector, albeit with a small lag.
Interestingly, the portfolio-level analysis contrasts with the results for individual equities, where multiple periods of 
structural breaks were observed.
Aggregating returns into a portfolio may have smoothed out smaller, company-specific disruptions, highlighting only the
most prominent systemic events.
Overall, the results emphasize that significant structural breaks in the healthcare market's portfolio-level returns are rare,
occurring primarily during periods of severe financial or economic crises.